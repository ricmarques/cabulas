\documentclass[portuguese,a4paper,12pt,onecolumn,fleqn]{article}

%fleqn, na linha acima, serve para as equações ficarem alinhadas à esquerda 
% quando conjugadas com blocos \begin{gather*}

\usepackage{a4wide}		% Para reduzir as margens
%\usepackage[latin1]{inputenc}
\usepackage[utf8]{inputenc}
\usepackage{textcomp}
\usepackage[portuguese]{babel}

%\usepackage[ansinew]{inputenc}
%\usepackage[T1]{fontenc}
\usepackage{amsmath}	% Para fórmulas matemáticas
\usepackage{amsthm}		% Para o environment de proof
\usepackage{amsfonts}	% Para escrever "espaços" matematicos , como sejam R^2 ou K
\newtheorem*{proposition}{Proposition}		% Para o environment de proposition. O "*" e' para NÃO numerar
%\usepackage{pgfplots}	% Para gráficos xy (cartesianos)
%\setlength{\mathindent}{0pt}	% Alinhar equações à esquerda (em vez de ficarem centradas)

 

\usepackage{parskip}	% Para ter linha em branco entre parágrafos

\usepackage{fancyhdr}	% Para definir cabeçalhos

\usepackage{hyperref}	% Para criar links, via \url{URL}


% Linha horizontal com espaco antes e depois:

% A usar o comando sugerido pelo utilizador egreg em:
% http://tex.stackexchange.com/questions/30805/vertical-space-between-horizontal-lines

\newcommand{\myline}{\par
  \kern3pt % space above the rules
  \hrule height 1.5pt
  \kern2pt % space between the rules
  \hrule height 0.8pt
  \kern3pt % space below the rules
}


%opening

\pagestyle{fancy}
\fancyhead[C]{Cábula Pré-Prova Presencial de ``Matemática Preparatória''\\(elaborada por Ricardo Dias Marques - Aluno nº 1100281 - v1.0, 5-mar-2016)}
\fancyfoot[C]{Pág. \thepage}

\title{Cábula Pré-Prova Presencial de \\``Matemática Preparatória''}

\author{1ª versão (v1.0) elaborada em 4-mar-2016 por:
\\\textbf{Ricardo Dias Marques}\\
Aluno nº 1100281 da Licenciatura em Informática\\
uab@ricmarques.net / 1100281@estudante.uab.pt}



\begin{document}

\maketitle

Este documento está a ser disponibilizado nos termos da licença ``Creative Commons" de 
``Attribution-NonCommercial-ShareAlike 4.0 International (CC BY-NC-SA 4.0)":\\
\url{http://creativecommons.org/licenses/by-nc-sa/4.0/}

É possível que este documento contenha erros, pelos quais o(s) autor(es) \textbf{não} poderá/poderão assumir qualquer responsabilidade.


\myline	% Linha horizontal separadora

% A usar blocos de gather para alinhar equações à esquerda:

\section*{Diversos}

\subsection*{Fórmula Quadrática / Fórmula Resolvente:}


\begin{gather*}
ax^2 + bx + c = 0 
\Leftrightarrow x = \frac{-b \pm \sqrt{b^2 - 4ac}}{2a}
\end{gather*}

FONTE: Pág. 13 do texto ``Prerequisitos\_Calculo-SITE.pdf"


\subsection*{Potência negativa:}

\begin{gather*}
a^{-b} = \frac{1}{a^b}
\end{gather*}

\subsection*{Relação entre raízes e potências}

\subsubsection*{Relação entre raiz quadrada e potência:}

\begin{gather*}
\sqrt{x}  = x^\frac{1}{2}
\end{gather*}

\subsubsection*{Relação genérica entre raízes e potências:}
\begin{gather*}
\sqrt[n]{a^m} = a^\frac{m}{n}
\end{gather*}

FONTE: Pág. 10 do texto ``Prerequisitos\_Calculo-SITE.pdf"

\myline	% Linha horizontal separadora

\section*{Fatoração}

\subsection*{Soma de quadrados:}
\begin{gather*}
(a^2 + b^2) = a^2 + 2ab + b^2
\end{gather*}

FONTE: Pág. 7 do texto ``Prerequisitos\_Calculo-SITE.pdf"

\subsection*{Diferença de quadrados:}
\begin{gather*}
(a^2 - b^2) = (a + b)(a - b)
\end{gather*}

FONTE: Pág. 11 do texto ``Prerequisitos\_Calculo-SITE.pdf"

\subsection*{Soma e diferença de cubos:}
\begin{gather*}
(a^3 + b^3) = (a + b)(a^2 - ab + b^2)
\\
\\
(a^3 - b^3) = (a - b)(a^2 + ab + b^2)
\end{gather*}

FONTE: Pág. 11 do texto ``Prerequisitos\_Calculo-SITE.pdf"

\myline	% Linha horizontal separadora

\section*{Logaritmos}

\begin{gather*}
\log_a 1 = 0
\\
\\
\log_a a = 1
\\
\\
\log_a (b \cdot c) = \log_a b + \log_a c
\\
\\
\log_a \frac{b}{c} = \log_a b - \log_a c
\\
\\
\log_a b^c = c \cdot \log_a b
\\
\\
\log_a b = \frac{\log_c b}{\log_c a}
\\
\\
a^{\log_a b} = b
\end{gather*}


FONTE: Pág. 10 do texto ``Prerequisitos\_Calculo-SITE.pdf"


\myline	% Linha horizontal separadora

\section*{Limites}

\textit{``Proposição 4. Sejam ($u_n$) e ($v_n$) duas sucessões convergentes e seja $c$ uma constante.}

\textit{Então:}

(a) $\lim\limits_{n \to +\infty} c = c$

(b) $\lim\limits_{n \to +\infty}(u_n + v_n ) = \lim\limits_{n \to +\infty}u_n + \lim\limits_{n \to +\infty}v_n$

(c) $\lim\limits_{n \to +\infty}(c \cdot u_n ) = c \cdot \lim\limits_{n \to +\infty}u_n$

(d) $\lim\limits_{n \to +\infty}(u_n \cdot v_n ) = (\lim\limits_{n \to +\infty}u_n) \cdot (\lim\limits_{n \to +\infty}v_n)$

(e)	$\lim\limits_{n \to +\infty}\frac{u_n}{v_n} = \frac{\lim\limits_{n \to +\infty}u_n}{\lim\limits_{n \to +\infty}v_n}$, 
desde que $\lim v_n \ne 0$

(f) $\lim\limits_{n \to +\infty}\sqrt[p]{u_n} = \sqrt[p]{\lim\limits_{n \to +\infty} u_n}$, desde que $p$ seja ímpar ou $u_n > 0$

FONTE: 

\myline	% Linha horizontal separadora


\section*{Derivadas}

\subsection*{Regras de derivação}

\subsubsection*{Definição de derivada:}

\begin{gather*}
f'(a) = \lim_{x \to a} \frac{f(x) - f(a)}{x - a}
\end{gather*}

FONTE: Pág. 1 do texto ``CQES\_derivadas-NOVO.pdf"


\subsubsection*{Derivada de uma constante:}

\begin{gather*}
c' = 0 
\end{gather*}
($c$ é um número real qualquer)

FONTE: Pág. 6 do texto ``CQES\_derivadas-NOVO.pdf"


\subsubsection*{Derivada da função identidade:}

\begin{gather*}
x' = 1
\end{gather*}

FONTE: Pág. 7 do texto ``CQES\_derivadas-NOVO.pdf"


\subsubsection*{Derivada da potência:}

\begin{gather*}
(x^n)' = n x^{n-1}
\end{gather*}

FONTE: Pág. 7 do texto ``CQES\_derivadas-NOVO.pdf"


\subsubsection*{Derivada da função exponencial:}

\begin{gather*}
(e^x)' = e^x
\end{gather*}

FONTE: Pág. 8 do texto ``CQES\_derivadas-NOVO.pdf"


\subsubsection*{Derivada da função logarítmica:}

\begin{gather*}
(\log x)' = \frac{1}{x}
\end{gather*}

FONTE: Pág. 8 do texto ``CQES\_derivadas-NOVO.pdf"


\subsubsection*{Derivada da função seno:}

\begin{gather*}
(\textrm{sen } x)' = \textrm{cos } x
\end{gather*}

FONTE: Pág. 8 do texto ``CQES\_derivadas-NOVO.pdf"


\subsubsection*{Derivada da função cosseno:}

\begin{gather*}
(\textrm{cos } x)' = - \textrm{sen } x
\end{gather*}

FONTE: Pág. 8 do texto ``CQES\_derivadas-NOVO.pdf"


\myline	% Linha horizontal separadora


\subsection*{Regras algébricas de derivação}

\subsubsection*{Derivada do produto de uma constante $c$ por uma função $f$:}

\begin{gather*}
(c \cdot f)' = c \cdot f'
\end{gather*}

FONTE: Pág. 8 do texto ``CQES\_derivadas-NOVO.pdf"


\subsubsection*{Derivada da soma (ou diferença) de duas funções $f$ e $g$:}

\begin{gather*}
(f + g)' = f' + g'
\end{gather*}

FONTE: Pág. 8 do texto ``CQES\_derivadas-NOVO.pdf"


\subsubsection*{Derivada do produto de duas funções $f$ e $g$:}

\begin{gather*}
(f \cdot g)' = f' \cdot g + f \cdot g'
\end{gather*}

FONTE: Pág. 8 do texto ``CQES\_derivadas-NOVO.pdf"


\subsubsection*{Derivada do quociente de duas funções $f$ e $g$:}

\begin{gather*}
\biggl(\frac{f}{g}\biggr)' = \frac{f' \cdot g - f \cdot g'}{g^2}
\end{gather*}

(nos pontos em que $g \ne 0$)

FONTE: Pág. 8 do texto ``CQES\_derivadas-NOVO.pdf"


\myline	% Linha horizontal separadora


\subsection*{Regras generalizadas de derivação}

Nas regras generalizadas constantes desta secção do formulário: $u = u(x)$

\subsubsection*{Derivada da potência:}

\begin{gather*}
(u^n)' = n u^{n-1} \cdot u'
\end{gather*}

FONTE: Pág. 16 do texto ``CQES\_derivadas-NOVO.pdf"


\subsubsection*{Derivada da função exponencial:}

\begin{gather*}
(e^u)' = u' e^u
\end{gather*}

FONTE: Pág. 16 do texto ``CQES\_derivadas-NOVO.pdf"


\subsubsection*{Derivada da função logarítmica:}

\begin{gather*}
(\log u)' = \frac{u'}{u}
\end{gather*}

FONTE: Pág. 16 do texto ``CQES\_derivadas-NOVO.pdf"


\subsubsection*{Derivada da função seno:}

\begin{gather*}
(\textrm{sen } u)' = u' \textrm{cos } u
\end{gather*}

FONTE: Pág. 16 do texto ``CQES\_derivadas-NOVO.pdf"


\subsubsection*{Derivada da função cosseno:}

\begin{gather*}
(\textrm{cos } u)' = -u' \textrm{sen } x
\end{gather*}

FONTE: Pág. 16 do texto ``CQES\_derivadas-NOVO.pdf"

\myline	% Linha horizontal separadora


\begin{center}
FIM
\end{center}

\end{document}

